\chapter{Systematic Constraint Handling}\label{chap:conhand}
\begin{overview}
  This chapter presents the proposed method of constraint handling based on the work of \citet{vinsonphd}.
  The method comprises:
  \begin{itemize}
    \item constraint checking for feasibility,
    \item quantification of operability clamping after constraint changes,
    \item the implementation of linear constraint in the commercial MPC structure, and
    \item constraint set fitting for reduction of the control problem.
  \end{itemize}
  A section is also devoted to disambiguating the language used to refer to constraints based on their sources.
\end{overview}

\section{Assumptions}
Certain assumptions are made about the systems investigated.
This serves to define the scope of the project and aids in the mathematical rigour of the methods.
\begin{itemize}
\item Steady-state models are used.
  These models result in real matrices and allow for strictly linear transformations of spaces.
\item Only linear constraints are considered.
  This allows for half-space geometry to be used and ensures convexity of intersections (along with the assumption of steady-state gain models).
\item The convexity of all spaces are assumed.
  This allows the use of minimum vertex descriptions.
\end{itemize}
Only a subset of the available model types are presented in this dissertation.
It is by no means implied that rigorous mathematical methods are only possible for this given subset.

Note that when reference is made to a ``constraint set'', all the half-spaces comprising that set as well as the feasible region (of the set) is implied.

\section{Constraint checking}
Commercial MPCs use only their set engineering limits to check the validity of constraints.
Along with the engineering constraints (which represent the ultimate bounds) the model should also be used to validate constraints. 
This is due to the constraints in the input and output space being interconnected via the process model.

\subsection{Feasibility}
Operating regions, such as the DOS, is usually specified using external requirements on the inputs or outputs.
The validity of such a space should be checked against the attainable regions of the corresponding inputs or outputs.
In this respect, determining the Operability Index provides a good measure of how valid a specified operating region is.

\subsection{Constraint changes}
The operator interfaces of commercial MPCs (figures~\ref{fig:ssrmpctoperator} and \ref{fig:ssdmcplusoperator}) allow operators to change constraints during the operation of processes.
The effects of constraint changes on their corresponding spaces are, however, neglected in commercial MPCs.
The reduction or increase in size of the available input space for a change in output constraints (or vice versa) should be checked.
Ideally a measure of clamping (or relaxing) of constraints should be supplied.

It is intuitive that constraint tightening in the output space has a corresponding tightening effect in the input space, but to which extent isn't immediately clear.
To quantify the tightening of the inputs, the OI of \citet{vinsonphd} can be used, although, this has the downside of remaining at a value of 1 when the DIS is already contained within the AIS, regardless of the size change of the DIS.
Another possibility is a ratio of the hypervolumes of the original attainable part of the DIS and the attainable part of the new DIS ($DISn$).
This clamping factor ($C$) is shown in equation~\ref{eq:inputclamp}.
\begin{equation}
  \label{eq:inputclamp}
  C = 1-\frac{\mu(AIS \cap DISn)}{\mu(AIS \cap DIS)}
\end{equation}
where $\mu$ is a function to calculate the hypervolume of a space.

\subsection{Setpoints}
The same argument for feasible constraints apply to setpoints.
The specification of setpoints should be checked to be within the attainable output space for a given available output space.

\section{Commercial MPC interfacing}
As discussed in section~\ref{sec:commercialmpc}, most commercial MPC packages only allow for high and low limits to be imposed on variables.

\subsection{Linear constraints}\label{sec:lincons}
For the case where constraints that are linear combinations of variables are present (e.g. $\beta_1 y_1+\beta_2 y_2\leq b_1$), they need to be reformatted to conform to the limited structure provided by commercial MPCs.
Adding an unmeasured variable to a system by adding rows to the process matrix is a common technique used with commercial MPCs.
This same technique can be used to impose linear constraints  on outputs or inputs.
The unmeasured variables added to the process model will have gain constants as determined by the coefficients in the linear constraint.

Considering a model $G$ for an $n\times m$ system ($n$ inputs, $m$ outputs) and the linear constraints 
\begin{align}
  \alpha \vect{u} &\leq b_u\notag\\ 
  \beta \vect{y} &\leq b_y \notag   
\end{align}
where $\alpha$ and $\beta$ are rows of $A_u$ and $A_y$ (equation~\ref{eq:gencons}) respectively, the following applies:
\begin{itemize}
\item For an input constraint a row is added to the process model and the new input, $u_{n+1}$ is defined thus:
\begin{equation}
  \label{eq:linconinput}
   u_{n+1} = \alpha
\end{equation}
Even though $u_{n+1}$ represents an input constraint, it is strictly speaking an additional output and the number of inputs to the system model remains unchanged. 
\item For an output constraint a row is added to the process model and the new output, $y_{m+1}$ is defined thus:
\begin{equation}
  \label{eq:linconoutput}
   y_{m+1} = (G'\times\vect{1}_{m\times 1})'\cdot\beta
\end{equation}
where $\vect{1}_{m\times 1}$ is a column vector of 1s of length $m$.
\end{itemize}
A high or low limit can now simply be applied to this newly added variable.
For the examples above, this is simply $u_{n+1}\leq b_u$ and $y_{m+1}\leq b_y$


\section{Constraint set fitting}
As shown by \citet{vinsonphd} -- by means of the Operability Index -- a larger operating region is advantageous for control.
The largest operating region (when considering the output space) is represented by the AOS and DOS intersection.
Ideally all the constraints specifying that intersection (the facets of the intersection) could be implemented in an MPC which would result in the maximum operating range for the controller.

Firstly, as the dimensions of the system increase the dimensions of the polytopes describing the input and output spaces increase as well.
This results in a high number of facets present in the intersections of spaces.
Furthermore, as the irregularity of the spaces increase the number of facets present in the intersections will also increase.
Each of these facets represent a constraint that can be added to the controller.

To add to this, the structure available for constraints in commercial MPCs require linear constraints to be expressed as unmeasured variables with high and low limits.
As mentioned in the preceding section, each linear constraint will add an additional variable along with its constraints to the system.

Therefore, although ideal, implementing the full description of a highly faceted AOS and DOS intersection will greatly increase the control problem.
This will have an adverse effect on the computational time of the controller.
The description of the controller will also suffer, as a large number of unmeasured variables (for the purpose of imposing constraints) can make housekeeping difficult.

The fitting of a constraint set (with fewer constraints) within the AOS and DOS intersection can be used to reduce the control problem.
The operating region of the smaller (fitted) constraint set will inevitably be less than that of the full intersection.
The fitting procedure therefore becomes a compromise between control problem size and the available operating region for the controller.

The sections that follow formally define the fitting problem and gives attention to the some important classes of set fitting and reduction.
Although the output space (AOS and DOS intersection) is used for examples and explanations the same applies to the input space.

\subsection{Problem formulation}
In mathematical terms, the problem consists of fitting the largest volume polytope (with a specified number of facets) within another polytope (with more facets than the fitted polytope).
Each half-space -- or the facets of a shape -- represents a constraint.
The constraint set can therefore be described by equation~\ref{eq:cset}, where $A$ is the half-plane slope matrix and $\vect{b}$ the offsets.
Although $\vect{x}$ is used in equation~\ref{eq:cset}, substitution with $\vect{u}$ or $\vect{y}$ for input or output constraints can be done without any loss of generality.
\begin{equation}
  \label{eq:cset}
  A\vect{x} \leq \vect{b}
\end{equation}
In $n$ dimensions, the general problem now becomes: 
\begin{itemize}
  \item given a polytope, $P_o$, having $l$ facets,
  \item fit a polytope, $P_i$, with $k$ facets (where $n+1 \leq k \le l$) within $P_o$, and
  \item maximise the volume of $P_i$.
\end{itemize} 
With the specification of the number of facets, the dimensions of $A$ and $\vect{b}$ for both $P_i$ and $P_o$ are fixed.
For $P_o$, $A$ is an $l \times n$ matrix and $\vect{b}$ is a $l \times 1$ vector.
For $P_i$, $A$ is an $k \times n$ matrix and $\vect{b}$ is a $k \times 1$ vector.

The fitting of an arbitrary shape can be expressed as a minimization problem, as shown in equation~\ref{eq:arbvolmin}.
\begin{align}
  \label{eq:arbvolmin}
    \min_{A,~\vect{b}}~~&\text{-volume}_{P_i}\\
    \text{s.t.}~~~~ &P_i \subset P_o \notag \\
                    &P_i \text{~and~} P_o \text{~convex} \notag
\end{align}
This seemingly trivial optimisation problem is not covered in this form in available literature.
A similar problem is that of optimised diamond cutting \citep{diamondcut} which maximises the volume of a diamond within a rough stone.
This method allows for rotation, translation and scaling, but uses a fixed shape for the diamond (i.e. all aspect ratios of the fitted shape stays constant).
Another well considered problem in literature is the packing problem, which involves packing the maximum number of shapes in a given shape (or container).
Again, the shapes considered in this problem are of fixed dimensions (and aspect ratios), but the largest dissimilarity is that a number of shapes are fitted, not just a single one.

\subsection{Set reduction}
Implementing all the constraints of the full AOS and DOS intersection does not require any fitting to be done.
The linear constraints only need to be converted to the high and low limit structure (section~\ref{sec:lincons}).

The fitting of constraint sets will decrease the operational space of the controller, but also decrease the size of the control problem.
The extent to which the control problem will be decreased depends on the number of constraints fitted and the type of fit.

\subsubsection{Intermediate fits}
As per equation~\ref{eq:arbvolmin}, a progressively lower number of constraints can be fitted within the intersection as arbitrary shapes.
The closer the number of constraints in fitted shape is to that of the original intersection, the larger the operational space of the controller will be.

To reduce the number of additional variables that need to be added to implement linear constraints, parallel sets can be used.
From equations~\ref{eq:linconinput} and \ref{eq:linconoutput} it is clear that the gains of variables added to the process model is only dependent on the half-space slope and the process model.
Therefore, parallel constraints only require one variable to be added to the process model.

Sets of parallel constraints can therefore be fitted, which will require one additional variable to be added to the process model per pair.
As the slopes of the parallel constraints are not fixed (for the minimisation problem) this will be referred to as free parallel fitting.
Enforcing these parallel constraints on the minimisation problem will result in a lower operational space for the controller when compared to an arbitrary shape with the same number of constraints.

The fitting of free parallel sets is presented as an option here, but not covered in this project.

\subsubsection{High and low limits}
The most common method of specifying constraints for commercial MPCs is using high and low limits.
Fitting the maximal volume set of high and low constraints within an intersection is a subset of free parallel fitting and has the following advantages: fitting results are directly applicable to commercial MPC packages, and no additional variables need to be defined.

The disadvantage of this method is a further decrease in the operational space of the controller when compared to free parallel fitting.

For the fitting of a rectangular constraint set (high and low limits), equation~\ref{eq:arbvolmin} reduces to equation~\ref{eq:recvolmin}.
In this description $\vect{b}$ is the only input variable (of size $2n \times 1$) and $I$ is an $n\times n$ identity matrix.
\begin{align}
  \label{eq:recvolmin}
    \min_{\vect{b}}~~&\text{-volume}_{P_i}\\
    \text{s.t.}~~~~ &A = \bpm I \\ -I \epm \notag \\
                    &P_i \subset P_o \notag \\
                    &P_i \text{~and~} P_o \text{~convex} \notag
\end{align}
where $n$ is the number of dimensions (variables) present in the initial set.

\subsubsection{Minimal fit}
A minimum value exists for the number of constraints that can be fitted into an \npoly.
This is due to the condition that the resulting set needs to be fully bounded and have a finite, non-zero volume.
For $n$ dimensions, $n$ points lie in a plane and result in a zero volume.
Therefore, at least $n+1$ points are needed to form a \npoly with a finite volume, with the resulting polytope being a simple polytope.
With the number of vertices at $n+1$, the minimum number of facets can be calculated as $n+1$ \citep{barnette}. 
Therefore, the smallest number of constraints that can be fitted into a set is $n+1$, where $n$ is the number of dimensions (variables) of the initial set.

\subsection{Fitting implementation}
The implementation of the fitting procedure is discussed in the code manual and program listing accompanying this document.
Therein, attention is also given to different problem formulations and the advantages and disadvantages thereof.

\section{Constraint types}
When constraints are considered, and specifically changes to them, the types of constraint present are important.
This is due to the direction of change allowed by each constraint type.
The following set of constraint types are proposed to disambiguate the specification of constraints.
These proposed types are based on the sources of constraints.
\begin{description}
  \item [Physical constraints] represent the physical limits of the system.
  Examples are tank levels, valve saturation limits and maximum heat duties.
  Allowed changes to these constraints are usually one-sided or to the inside of a range.
  \item [Quality constraints] are concerned with product quality, these constraints are typically determined by external requirements or standards.
  Strictly speaking, changes to these constraints are allowed in any direction.
  However, due to the economic (e.g. product purity) and safety (e.g. emission limits) consequences of changes, these limits are only changed by those with granted access.
  \item [Operational constraints] are the innermost constraint set used by the MPC controller.
  These constraints are typically changed using the operator interfaces (e.g. figures~\ref{fig:ssrmpctoperator} and \ref{fig:ssdmcplusoperator}) and are contained within their respective variables' physical and quality constraints.
  \item [Optimisation constraints] are those constraints that are not critical for the control of the plant, but rather the performance and efficiency of the plant as determined by external cost functions.
  Typically these external cost functions will represent a type of saving, e.g. energy savings or product give-away reduction.
\end{description}

The specification of spaces, in particular the AIS and the DOS, usually consists of a combination of the constraint types mentioned above.
Classifying constraints in this unambiguous manner will further the understanding of a system and help identify which constraints can be modified to change the size or shape of a space.

Commercial MPCs do not typically distinguish between different types of constraints.
The constraint types present in RMPCT (section~\ref{sec:rmpctcons}) and DMCplus (section~\ref{sec:dmcpluscons}) are listed below and their connection to the proposed constraint types discussed.
\begin{description}
  \item [Engineering limits] are typically based on a combination of physical constraints and quality constraints.
  The specification of the engineering limits should ideally be a subset of the physical constraints and the AOS and AIS of the process.
  One method would be to specify high and low limits on the outputs determined by their maximum values in the AOS.
  Along with a safety factor -- dependant on model uncertainty -- the intersection of these spaces would provide a better set of engineering limits.
  \item [Validity limits] (DMCplus) are equal to or outside of the physical constraints of a variable.
\end{description}



% Local Variables:
% TeX-master: "AHC_thesis"
% End:
