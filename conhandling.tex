\chapter{Constraint Handling}\label{chap:conhand}
\begin{overview}
  [Constraint handling overview...]
\end{overview}

%%% OTHER SECTIONS - LAYOUT
\section{Interactivity}
\citet{vinsonphd} presents a list of issues that limit the performace of MPCs
(with specific attention to DMCplus) and proposes methods (using the 
operability index) to solve these problems. Most of these methods are directly
applicable to the proposed contraint handling method of this document. The
issues identified by \citet{vinsonphd} as well as additions are shown in table
\ref{tab:mpcissues}.
\begin{landscape}
\begin{table}[htbp]
  \caption[Methods for improving the working and understanding of MPCs]
    {Methods for improving the working and understanding of MPCs}
  \label{tab:mpcissues}
  \centering
  \begin{tabular}{p{8cm} p{8cm} p{8cm}}
    \toprule
    Issues hampering MPC potential & Current solution & Proposed solution \\
    \midrule
    Processes with considerable noise cause targets to move with each
      controller execution.
      & Limit MV stepsize or filter CVs. Both of these have a stabalizing effect
        but result in a more sluggish controller. These limits may also cause
        the controller to not reach optimum solutions.
      & Using the OI and the associated output spaces, the possible outputs of
        a (newly) limited input space is clearly shown.\\
    Constraints on MV can cause CVs to not reach their maximum/minimum
    contraints. These CV constraints usually correspond with economically
    optimal solutions. 
      & Increase/Adjust the MV limits. 
      & The spaces associated with the OI would make it clear to what extent
        MV limits can be adjusted and how they would affect the output space.\\
    Disturbances can cause CVs to violate constraints. In some cases, the
    controller might give up on a constraint and continue (now using a subset
    of the CVs).
      & Possible solutions include; model re-identification (in the case of an
        inconsistency), fixing the source of the disturbance or adding a DV to
        account for the disturbance.
      & Due to the fixed constraints on MVs and CVs the resulting AIS and DOS
        can be used to calculate the controller's ability to reject a
        disturbance. A measure of controller robustness can be derived from
        this.\\
    Move Suppression factors for MVs and Equal Concern Errors (give-ups) for 
    CVs are used to tune MPCs for dynamic performance.
      & The choice of these factors are usually qualitative and according to 
        guidelines (and validated via offline simulations).
      & The OI (and its associated spaces) can provide guidance in determining
        move suppression factors and give-ups. The difficulty in controlling
        a CV at a given constraint can be derived from the OI. How far a CV
        constraint can be moved (given the MV constraints) can also be
        determined from the OI.\\
    \bottomrule
  \end{tabular}
\end{table}
\end{landscape}