\chapter{Systematic Constraint Handling}\label{chap:conhand}
\begin{overview}
  This chapter gives an overview of the proposed method of constraint handling based on the work of \citet{vinsonphd}.
  General application of the OI for improved MPC operation is also reviewed.
\end{overview}

\section{Constraint checking}
Commercial MPCs use only their set engineering limits to check the validity of constraints.
Constraints in the input and output space are interconnected via the process model.
Along with the engineering constraints (which represent the ultimate bounds) the model should be used to validate constraints. 
\subsection{Feasibility}
Operating regions, such as the DOS, is usually specified using external requirements on the inputs/outputs.
The validity of such a space should be checked against the attainable regions of the corresponding inputs/outputs.
In this respect, determining the Operability Index provides a good measure of how valid a specified operating region is.
\subsection{Constraint changes}
Changing constraints during process operation is a common procedure.
The effects of constraint changes on their corresponding input/output spaces are, however, neglected in commercial MPCs.
Checking the reduction/increase in size of available input space for a change in output constraints (or vice versa) should be done.
Ideally a measure of clamping/relaxing of constraints should be supplied.
\subsection{Setpoints}
The same argument for feasible constraints apply to setpoints.
The specification of setpoints should be checked to be within the attainable output space for a given available output space.

\section{Constraint set fitting}
As shown by \citet{vinsonphd} -- by means of the Operability Index -- a larger operating region is advantageous for control.
Fitting a constraint set within the AOS/DOS intersection (whilst maximising its volume) will increase the OI and yield feasible constraints.
\subsection{Set reduction}
When considering the AOS/DOS intersection (and all the constraints specifying it), the control problem can be reduced by fitting a set with less constraints therein.
A constraint set with less constraints than the original AOS/DOS intersection can be fitted and its volume (i.e. operating region) maximised. 
\subsection{High/low limits}
The most common method of specifying constraints for commercial MPCs is using high/low limits.
Fitting the maximal volume set of high/low constraints within the AOS/DOS intersection increases the OI, yields constraints and an operating region that is completely feasible, and is directly applicable to commercial MPC packages.
\subsection{Constraint types}
The engineering constraints specified for processes are typically based on physical limits of process equipment.
The specification of the engineering limits should ideally be a subset of the physical limits and the AOS of the process.
One method would be to specify high/low limits on the outputs determined by their maximum values in the AOS.
Along with a safety factor -- dependant on model uncertainty -- the intersection of these spaces would provide a better set of engineering limits.

\section{Commercial MPC interfacing}
\subsection{Linear constraints}
For the case where constraints that are linear combinations of variable are present, they need to be reformatted to conform to the high/low limit structure used by commercial MPCs.
Linear constraints should therefore be added as unmeasured variables to the process model, with gain constants as determined by the coefficients in the linear constraint.
A high/low limit can now simply be applied to this newly added variable.

\section{General application of the OI}
\citet{vinsonphd} presents a list of additional issues that limit the performance of MPCs (with specific attention to DMCplus) and proposes methods (using the operability index) to solve these problems. 
These methods are indirectly applicable to the proposed constraint handling method of this document and are presented here for that reason. 
The issues identified by \citet{vinsonphd} are shown in table~\ref{tab:mpcissues}.
\begin{landscape}
\hfill
\begin{table}[htbp]
  \caption[Methods for improving the working and understanding of MPCs]
    {Methods for improving the working and understanding of MPCs \citep{vinsonphd}}
  \label{tab:mpcissues}
  \centering
  \begin{tabular}{p{8cm} p{8cm} p{8cm}}
    \toprule
    Issues hampering MPC potential & Current solution & Proposed solution \\
    \midrule
Processes with considerable noise cause targets to move with each controller execution.
& Limit MV stepsize or filter CVs. 
Both of these have a stabilizing effect but result in a more sluggish controller. 
These limits may also cause the controller to not reach optimum solutions.
& Using the OI and the associated output spaces, the possible outputs of a (newly) limited input space is clearly shown.\\[1.3ex]
    
Disturbances can cause CVs to violate constraints. 
In some cases, the controller might give up on a constraint and continue (now using a subset of the CVs).
& Possible solutions include; model re-identification (in the case of an inconsistency), fixing the source of the disturbance or adding a DV to account for the disturbance.
& Due to the fixed constraints on MVs and CVs the resulting AIS and DOS can be used to calculate the controller's ability to reject a disturbance. 
A measure of controller robustness can be derived from this.\\[1.3ex]
    
Move Suppression factors for MVs and Equal Concern Errors (give-ups) for CVs are used to tune MPCs for dynamic performance.
& The choice of these factors are usually qualitative and according to guidelines (and validated via offline simulations).
& The OI (and its associated spaces) can provide guidance in determining move suppression factors and give-ups. 
The difficulty in controlling a CV at a given constraint can be derived from the OI. 
How far a CV constraint can be moved (given the MV constraints) can also be determined from the OI.\\
    \bottomrule
  \end{tabular}
\end{table}
\hfill
\end{landscape}

% Local Variables:
% TeX-master: "AHC_thesis"
% End: