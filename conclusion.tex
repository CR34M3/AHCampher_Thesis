\chapter{Conclusions and Recommendations}\label{chap:conclusion}
\begin{overview}
  This chapter summarises the conclusions drawn from this project and makes recommendations for further study.
\end{overview}

\section{Conclusions}
The following conclusions can be drawn based on the results of the proposed method of systematic constraint handling:
\begin{itemize}
  \item Specification of unrealistic constraints leads to a low Operability Index and unattainable expectations of both the inputs and outputs of a process.
  \item If constraints are checked via the process model, constraint sets that are not attainable can be identified. Also, checking the changes made to constraints (via the process model) allow for a measure of clamping to be determined.
  \item Constraints that are linear combinations of variables can be implemented in commercial MPCs by adding unmeasured variables to the process and imposing high or low constraints thereon.
This procedure improves the operating region of the controller but increases the size of the control problem.
  \item The results obtained by the method are readily applicable to commercial MPCs via user interfaces or as external programs.
\end{itemize}

From the process of fitting constraint sets, these additional conclusions can be drawn:
\begin{itemize}
\item The size of a constraint set can be reduced by fitting a smaller set within, while maximising the operating region of the fitted set.
\item A more rigorous formulation is obtained when the optimisation problem is expressed in terms of facets rather than vertices, but this increases the solution complexity.
\item The constrained problem formulation resulted in faster solution times when compared to the unconstrained case.
\item The accuracy of fitting for arbitrary sets yielded the following results:
  \begin{itemize}
  \item 65.8\% of fitting results were within 10\% of the optimum solution for set fitting in 2 dimensions.
  This increased to 100\% within 3\% of the optimum when using a multi-start approach.
  \item 0\% of fitting results were within 10\% of the optimum solution for set fitting in 3 dimensions.
  22.8\% was the minimum error obtained for the tests.
  This low accuracy is ascribed to ill conditioning of the problem and increased superfluous degrees of freedom.
  \end{itemize}
\end{itemize}

\section{Recommended future research}
The conclusions made and work presented in this dissertation opens a few avenues for further improvement and research.
\subsection{Systematic constraint handling}
At present the proposed method of systematic constraint handling consists only of routines to do calculations and support only a small subset of possible models.
The following would be attractive additions to the method:
\begin{itemize}
  \item Support for more model types.
  \item A graphical user interface to bind all the routines into an easily accessible tool.
  \item An improved method of data representation for higher order systems, as graphical representation becomes difficult in more than 3 dimensions. 
\end{itemize}
\subsection{Constraint set fitting}
The process of constraint set fitting should be improved upon to increase its accuracy for higher dimensional systems.

Also, the incorporation of additional functions into the fitting procedure would improve the output of this process.
Fitting done with respect to the volume integral of (for example) economic or sensitivity functions would highlight regions of economic or control concern.

\subsection{The Operability Index}
While not strictly speaking part of this project, an extension of the Operability Index and its associated spaces is proposed.
The inclusion of dynamics in the models should produce a transient space around the original AOS.
The information obtained from this space could be valuable in determining optimal and realistic dynamic constraints for MPC packages, e.g. the funnels of RMPCT.
This transient space could also be used to help determine optimal move paths.

% Local Variables:
% TeX-master: "AHC_thesis"
% End:
