\chapter{Case Studies}\label{chap:casestudies}
\begin{overview}
  The case studies investigated in this project are presented in this chapter.
  Details are given on design, models and operating conditions of the systems. 
\end{overview}

\section{Case studies}
To test the method of systematic constraint handling proposed in chapter~\ref{chap:conhand} the following physical systems were investigated.
Both rigs are from the Process Modelling and Control laboratory of the University of Pretoria.

\subsection{Level and flow rig}
\subsubsection{System description}
The level and flow rig consists of two control valves, a measured flow (to one control valve) and a measured level.
Figure~\ref{fig:flowphoto} shows a photograph of the rig along with its process flow diagram.
\begin{figure}[htbp]
  \centering
    \includegraphics[width=7.0cm]{graph/flowphoto.JPG}
    %\qquad
%    \includegraphics[width=7.8cm]{graph/flowpfd}
    \scalebox{1}{% Graphic for TeX using PGF
% Title: /home/andre/GIT Repos/AHCampher_thesis/diagrams/flowpfd.dia
% Creator: Dia v0.97.1
% CreationDate: Wed Sep  7 23:09:35 2011
% For: andre
% \usepackage{tikz}
% The following commands are not supported in PSTricks at present
% We define them conditionally, so when they are implemented,
% this pgf file will use them.
\ifx\du\undefined
  \newlength{\du}
\fi
\setlength{\du}{15\unitlength}
\begin{tikzpicture}
\pgftransformxscale{1.000000}
\pgftransformyscale{-1.000000}
\definecolor{dialinecolor}{rgb}{0.000000, 0.000000, 0.000000}
\pgfsetstrokecolor{dialinecolor}
\definecolor{dialinecolor}{rgb}{1.000000, 1.000000, 1.000000}
\pgfsetfillcolor{dialinecolor}
\pgfsetlinewidth{0.100000\du}
\pgfsetdash{}{0pt}
\pgfsetdash{}{0pt}
\pgfsetbuttcap
\pgfsetmiterjoin
\pgfsetlinewidth{0.100000\du}
\pgfsetbuttcap
\pgfsetmiterjoin
\pgfsetdash{}{0pt}
\definecolor{dialinecolor}{rgb}{1.000000, 1.000000, 1.000000}
\pgfsetfillcolor{dialinecolor}
\pgfpathellipse{\pgfpoint{4.275000\du}{6.750000\du}}{\pgfpoint{0.500000\du}{0\du}}{\pgfpoint{0\du}{0.500000\du}}
\pgfusepath{fill}
\definecolor{dialinecolor}{rgb}{0.000000, 0.000000, 0.000000}
\pgfsetstrokecolor{dialinecolor}
\pgfpathellipse{\pgfpoint{4.275000\du}{6.750000\du}}{\pgfpoint{0.500000\du}{0\du}}{\pgfpoint{0\du}{0.500000\du}}
\pgfusepath{stroke}
\pgfsetbuttcap
\pgfsetmiterjoin
\pgfsetdash{}{0pt}
\definecolor{dialinecolor}{rgb}{1.000000, 1.000000, 1.000000}
\pgfsetfillcolor{dialinecolor}
\fill (3.921447\du,7.103550\du)--(3.775000\du,7.333333\du)--(4.775000\du,7.333333\du)--(4.628550\du,7.103550\du)--cycle;
\pgfsetbuttcap
\pgfsetmiterjoin
\pgfsetdash{}{0pt}
\definecolor{dialinecolor}{rgb}{0.000000, 0.000000, 0.000000}
\pgfsetstrokecolor{dialinecolor}
\pgfpathellipse{\pgfpoint{4.275000\du}{6.750000\du}}{\pgfpoint{0.500000\du}{0\du}}{\pgfpoint{0\du}{0.500000\du}}
\pgfusepath{stroke}
\pgfsetbuttcap
\pgfsetmiterjoin
\pgfsetdash{}{0pt}
\definecolor{dialinecolor}{rgb}{0.000000, 0.000000, 0.000000}
\pgfsetstrokecolor{dialinecolor}
\draw (3.921447\du,7.103550\du)--(3.775000\du,7.333333\du)--(4.775000\du,7.333333\du)--(4.628550\du,7.103550\du);
\pgfsetlinewidth{0.100000\du}
\pgfsetdash{}{0pt}
\pgfsetdash{}{0pt}
\pgfsetbuttcap
\pgfsetmiterjoin
\pgfsetlinewidth{0.100000\du}
\pgfsetbuttcap
\pgfsetmiterjoin
\pgfsetdash{}{0pt}
\definecolor{dialinecolor}{rgb}{1.000000, 1.000000, 1.000000}
\pgfsetfillcolor{dialinecolor}
\pgfpathmoveto{\pgfpoint{8.187500\du}{3.189252\du}}
\pgfpathcurveto{\pgfpoint{8.187500\du}{3.000000\du}}{\pgfpoint{8.562500\du}{3.000000\du}}{\pgfpoint{8.562500\du}{3.189252\du}}
\pgfpathlineto{\pgfpoint{8.187500\du}{3.189252\du}}
\pgfusepath{fill}
\pgfsetbuttcap
\pgfsetmiterjoin
\pgfsetdash{}{0pt}
\definecolor{dialinecolor}{rgb}{1.000000, 1.000000, 1.000000}
\pgfsetfillcolor{dialinecolor}
\fill (8.000000\du,3.189252\du)--(8.750000\du,3.750000\du)--(8.750000\du,3.189252\du)--(8.000000\du,3.750000\du)--cycle;
\pgfsetbuttcap
\pgfsetmiterjoin
\pgfsetdash{}{0pt}
\definecolor{dialinecolor}{rgb}{0.000000, 0.000000, 0.000000}
\pgfsetstrokecolor{dialinecolor}
\draw (8.000000\du,3.189252\du)--(8.750000\du,3.750000\du)--(8.750000\du,3.189252\du)--(8.000000\du,3.750000\du)--cycle;
\pgfsetbuttcap
\pgfsetmiterjoin
\pgfsetdash{}{0pt}
\definecolor{dialinecolor}{rgb}{0.000000, 0.000000, 0.000000}
\pgfsetstrokecolor{dialinecolor}
\draw (8.375000\du,3.469626\du)--(8.375000\du,3.189252\du);
\pgfsetbuttcap
\pgfsetmiterjoin
\pgfsetdash{}{0pt}
\definecolor{dialinecolor}{rgb}{0.000000, 0.000000, 0.000000}
\pgfsetstrokecolor{dialinecolor}
\draw (8.187500\du,3.189252\du)--(8.562500\du,3.189252\du);
\pgfsetbuttcap
\pgfsetmiterjoin
\pgfsetdash{}{0pt}
\definecolor{dialinecolor}{rgb}{0.000000, 0.000000, 0.000000}
\pgfsetstrokecolor{dialinecolor}
\pgfpathmoveto{\pgfpoint{8.187500\du}{3.189252\du}}
\pgfpathcurveto{\pgfpoint{8.187500\du}{3.000000\du}}{\pgfpoint{8.562500\du}{3.000000\du}}{\pgfpoint{8.562500\du}{3.189252\du}}
\pgfusepath{stroke}
\pgfsetlinewidth{0.100000\du}
\pgfsetdash{}{0pt}
\pgfsetdash{}{0pt}
\pgfsetbuttcap
\pgfsetmiterjoin
\pgfsetlinewidth{0.100000\du}
\pgfsetbuttcap
\pgfsetmiterjoin
\pgfsetdash{}{0pt}
\definecolor{dialinecolor}{rgb}{1.000000, 1.000000, 1.000000}
\pgfsetfillcolor{dialinecolor}
\pgfpathmoveto{\pgfpoint{8.187500\du}{5.939252\du}}
\pgfpathcurveto{\pgfpoint{8.187500\du}{5.750000\du}}{\pgfpoint{8.562500\du}{5.750000\du}}{\pgfpoint{8.562500\du}{5.939252\du}}
\pgfpathlineto{\pgfpoint{8.187500\du}{5.939252\du}}
\pgfusepath{fill}
\pgfsetbuttcap
\pgfsetmiterjoin
\pgfsetdash{}{0pt}
\definecolor{dialinecolor}{rgb}{1.000000, 1.000000, 1.000000}
\pgfsetfillcolor{dialinecolor}
\fill (8.000000\du,5.939252\du)--(8.750000\du,6.500000\du)--(8.750000\du,5.939252\du)--(8.000000\du,6.500000\du)--cycle;
\pgfsetbuttcap
\pgfsetmiterjoin
\pgfsetdash{}{0pt}
\definecolor{dialinecolor}{rgb}{0.000000, 0.000000, 0.000000}
\pgfsetstrokecolor{dialinecolor}
\draw (8.000000\du,5.939252\du)--(8.750000\du,6.500000\du)--(8.750000\du,5.939252\du)--(8.000000\du,6.500000\du)--cycle;
\pgfsetbuttcap
\pgfsetmiterjoin
\pgfsetdash{}{0pt}
\definecolor{dialinecolor}{rgb}{0.000000, 0.000000, 0.000000}
\pgfsetstrokecolor{dialinecolor}
\draw (8.375000\du,6.219626\du)--(8.375000\du,5.939252\du);
\pgfsetbuttcap
\pgfsetmiterjoin
\pgfsetdash{}{0pt}
\definecolor{dialinecolor}{rgb}{0.000000, 0.000000, 0.000000}
\pgfsetstrokecolor{dialinecolor}
\draw (8.187500\du,5.939252\du)--(8.562500\du,5.939252\du);
\pgfsetbuttcap
\pgfsetmiterjoin
\pgfsetdash{}{0pt}
\definecolor{dialinecolor}{rgb}{0.000000, 0.000000, 0.000000}
\pgfsetstrokecolor{dialinecolor}
\pgfpathmoveto{\pgfpoint{8.187500\du}{5.939252\du}}
\pgfpathcurveto{\pgfpoint{8.187500\du}{5.750000\du}}{\pgfpoint{8.562500\du}{5.750000\du}}{\pgfpoint{8.562500\du}{5.939252\du}}
\pgfusepath{stroke}
\pgfsetlinewidth{0.100000\du}
\pgfsetdash{}{0pt}
\pgfsetdash{}{0pt}
\pgfsetbuttcap
\pgfsetmiterjoin
\pgfsetlinewidth{0.100000\du}
\pgfsetbuttcap
\pgfsetmiterjoin
\pgfsetdash{}{0pt}
\definecolor{dialinecolor}{rgb}{1.000000, 1.000000, 1.000000}
\pgfsetfillcolor{dialinecolor}
\fill (16.150000\du,5.800000\du)--(16.150000\du,8.800000\du)--(18.650000\du,8.800000\du)--(18.650000\du,5.800000\du)--cycle;
\pgfsetbuttcap
\pgfsetmiterjoin
\pgfsetdash{}{0pt}
\definecolor{dialinecolor}{rgb}{0.000000, 0.000000, 0.000000}
\pgfsetstrokecolor{dialinecolor}
\draw (16.150000\du,5.800000\du)--(16.150000\du,8.800000\du)--(18.650000\du,8.800000\du)--(18.650000\du,5.800000\du);
% setfont left to latex
\definecolor{dialinecolor}{rgb}{0.000000, 0.000000, 0.000000}
\pgfsetstrokecolor{dialinecolor}
\node[anchor=west] at (16.350000\du,7.550000\du){Feed};
\pgfsetlinewidth{0.100000\du}
\pgfsetdash{}{0pt}
\pgfsetdash{}{0pt}
\pgfsetmiterjoin
\definecolor{dialinecolor}{rgb}{1.000000, 1.000000, 1.000000}
\pgfsetfillcolor{dialinecolor}
\fill (10.000000\du,3.250000\du)--(10.000000\du,3.750000\du)--(11.750000\du,3.750000\du)--(11.750000\du,3.250000\du)--cycle;
\definecolor{dialinecolor}{rgb}{0.000000, 0.000000, 0.000000}
\pgfsetstrokecolor{dialinecolor}
\draw (10.000000\du,3.250000\du)--(10.000000\du,3.750000\du)--(11.750000\du,3.750000\du)--(11.750000\du,3.250000\du)--cycle;
\pgfsetlinewidth{0.100000\du}
\pgfsetdash{}{0pt}
\pgfsetdash{}{0pt}
\pgfsetbuttcap
\pgfsetmiterjoin
\pgfsetlinewidth{0.100000\du}
\pgfsetbuttcap
\pgfsetmiterjoin
\pgfsetdash{}{0pt}
\definecolor{dialinecolor}{rgb}{1.000000, 1.000000, 1.000000}
\pgfsetfillcolor{dialinecolor}
\fill (12.750000\du,2.750000\du)--(12.750000\du,5.500000\du)--(14.750000\du,5.500000\du)--(14.750000\du,2.750000\du)--cycle;
\pgfsetbuttcap
\pgfsetmiterjoin
\pgfsetdash{}{0pt}
\definecolor{dialinecolor}{rgb}{0.000000, 0.000000, 0.000000}
\pgfsetstrokecolor{dialinecolor}
\draw (12.750000\du,2.750000\du)--(12.750000\du,5.500000\du)--(14.750000\du,5.500000\du)--(14.750000\du,2.750000\du);
\pgfsetlinewidth{0.100000\du}
\pgfsetdash{}{0pt}
\pgfsetdash{}{0pt}
\pgfsetbuttcap
{
\definecolor{dialinecolor}{rgb}{0.000000, 0.000000, 0.000000}
\pgfsetfillcolor{dialinecolor}
% was here!!!
\pgfsetarrowsend{stealth}
\definecolor{dialinecolor}{rgb}{0.000000, 0.000000, 0.000000}
\pgfsetstrokecolor{dialinecolor}
\draw (4.275000\du,6.250000\du)--(8.000000\du,6.250000\du);
}
\pgfsetlinewidth{0.100000\du}
\pgfsetdash{}{0pt}
\pgfsetdash{}{0pt}
\pgfsetmiterjoin
\pgfsetbuttcap
{
\definecolor{dialinecolor}{rgb}{0.000000, 0.000000, 0.000000}
\pgfsetfillcolor{dialinecolor}
% was here!!!
\pgfsetarrowsend{stealth}
{\pgfsetcornersarced{\pgfpoint{0.000000\du}{0.000000\du}}\definecolor{dialinecolor}{rgb}{0.000000, 0.000000, 0.000000}
\pgfsetstrokecolor{dialinecolor}
\draw (6.250000\du,6.250000\du)--(6.250000\du,6.250000\du)--(6.250000\du,3.469630\du)--(8.000000\du,3.469630\du);
}}
\pgfsetlinewidth{0.100000\du}
\pgfsetdash{}{0pt}
\pgfsetdash{}{0pt}
\pgfsetbuttcap
{
\definecolor{dialinecolor}{rgb}{0.000000, 0.000000, 0.000000}
\pgfsetfillcolor{dialinecolor}
% was here!!!
\pgfsetarrowsend{stealth}
\definecolor{dialinecolor}{rgb}{0.000000, 0.000000, 0.000000}
\pgfsetstrokecolor{dialinecolor}
\draw (8.750000\du,3.500000\du)--(10.000000\du,3.500000\du);
}
\pgfsetlinewidth{0.100000\du}
\pgfsetdash{}{0pt}
\pgfsetdash{}{0pt}
\pgfsetmiterjoin
\pgfsetbuttcap
{
\definecolor{dialinecolor}{rgb}{0.000000, 0.000000, 0.000000}
\pgfsetfillcolor{dialinecolor}
% was here!!!
\pgfsetarrowsend{stealth}
{\pgfsetcornersarced{\pgfpoint{0.000000\du}{0.000000\du}}\definecolor{dialinecolor}{rgb}{0.000000, 0.000000, 0.000000}
\pgfsetstrokecolor{dialinecolor}
\draw (8.750000\du,6.219626\du)--(10.750000\du,6.219626\du)--(10.750000\du,4.125000\du)--(12.750000\du,4.125000\du);
}}
\pgfsetlinewidth{0.100000\du}
\pgfsetdash{}{0pt}
\pgfsetdash{}{0pt}
\pgfsetbuttcap
{
\definecolor{dialinecolor}{rgb}{0.000000, 0.000000, 0.000000}
\pgfsetfillcolor{dialinecolor}
% was here!!!
\pgfsetarrowsend{stealth}
\definecolor{dialinecolor}{rgb}{0.000000, 0.000000, 0.000000}
\pgfsetstrokecolor{dialinecolor}
\draw (11.750000\du,3.500000\du)--(12.750000\du,3.500000\du);
}
\pgfsetlinewidth{0.100000\du}
\pgfsetdash{}{0pt}
\pgfsetdash{}{0pt}
\pgfsetmiterjoin
\pgfsetbuttcap
{
\definecolor{dialinecolor}{rgb}{0.000000, 0.000000, 0.000000}
\pgfsetfillcolor{dialinecolor}
% was here!!!
\pgfsetarrowsend{stealth}
{\pgfsetcornersarced{\pgfpoint{0.000000\du}{0.000000\du}}\definecolor{dialinecolor}{rgb}{0.000000, 0.000000, 0.000000}
\pgfsetstrokecolor{dialinecolor}
\draw (14.750000\du,4.812500\du)--(15.748500\du,4.812500\du)--(15.748500\du,4.812500\du)--(17.414100\du,4.812500\du)--(17.414100\du,5.675000\du);
}}
\pgfsetlinewidth{0.100000\du}
\pgfsetdash{}{0pt}
\pgfsetdash{}{0pt}
\pgfsetbuttcap
\pgfsetmiterjoin
\pgfsetlinewidth{0.100000\du}
\pgfsetbuttcap
\pgfsetmiterjoin
\pgfsetdash{}{0pt}
\definecolor{dialinecolor}{rgb}{1.000000, 1.000000, 1.000000}
\pgfsetfillcolor{dialinecolor}
\pgfpathellipse{\pgfpoint{7.029370\du}{7.191870\du}}{\pgfpoint{0.550000\du}{0\du}}{\pgfpoint{0\du}{0.550000\du}}
\pgfusepath{fill}
\definecolor{dialinecolor}{rgb}{0.000000, 0.000000, 0.000000}
\pgfsetstrokecolor{dialinecolor}
\pgfpathellipse{\pgfpoint{7.029370\du}{7.191870\du}}{\pgfpoint{0.550000\du}{0\du}}{\pgfpoint{0\du}{0.550000\du}}
\pgfusepath{stroke}
% setfont left to latex
\definecolor{dialinecolor}{rgb}{0.000000, 0.000000, 0.000000}
\pgfsetstrokecolor{dialinecolor}
\node at (7.029370\du,7.191870\du){$F$};
\pgfsetlinewidth{0.100000\du}
\pgfsetdash{}{0pt}
\pgfsetdash{}{0pt}
\pgfsetbuttcap
\pgfsetmiterjoin
\pgfsetlinewidth{0.100000\du}
\pgfsetbuttcap
\pgfsetmiterjoin
\pgfsetdash{}{0pt}
\definecolor{dialinecolor}{rgb}{1.000000, 1.000000, 1.000000}
\pgfsetfillcolor{dialinecolor}
\pgfpathellipse{\pgfpoint{13.321300\du}{1.850000\du}}{\pgfpoint{0.550000\du}{0\du}}{\pgfpoint{0\du}{0.550000\du}}
\pgfusepath{fill}
\definecolor{dialinecolor}{rgb}{0.000000, 0.000000, 0.000000}
\pgfsetstrokecolor{dialinecolor}
\pgfpathellipse{\pgfpoint{13.321300\du}{1.850000\du}}{\pgfpoint{0.550000\du}{0\du}}{\pgfpoint{0\du}{0.550000\du}}
\pgfusepath{stroke}
% setfont left to latex
\definecolor{dialinecolor}{rgb}{0.000000, 0.000000, 0.000000}
\pgfsetstrokecolor{dialinecolor}
\node at (13.321300\du,1.850000\du){$L$};
\pgfsetlinewidth{0.050000\du}
\pgfsetdash{}{0pt}
\pgfsetdash{}{0pt}
\pgfsetbuttcap
{
\definecolor{dialinecolor}{rgb}{0.000000, 0.000000, 0.000000}
\pgfsetfillcolor{dialinecolor}
% was here!!!
\definecolor{dialinecolor}{rgb}{0.000000, 0.000000, 0.000000}
\pgfsetstrokecolor{dialinecolor}
\draw (7.000000\du,6.250000\du)--(7.010655\du,6.591704\du);
}
\pgfsetlinewidth{0.050000\du}
\pgfsetdash{}{0pt}
\pgfsetdash{}{0pt}
\pgfsetbuttcap
{
\definecolor{dialinecolor}{rgb}{0.000000, 0.000000, 0.000000}
\pgfsetfillcolor{dialinecolor}
% was here!!!
\definecolor{dialinecolor}{rgb}{0.000000, 0.000000, 0.000000}
\pgfsetstrokecolor{dialinecolor}
\draw (13.325000\du,2.406250\du)--(13.325000\du,3.156250\du);
}
% setfont left to latex
\definecolor{dialinecolor}{rgb}{0.000000, 0.000000, 0.000000}
\pgfsetstrokecolor{dialinecolor}
\node[anchor=west] at (8.000000\du,2.750000\du){$x_1$};
% setfont left to latex
\definecolor{dialinecolor}{rgb}{0.000000, 0.000000, 0.000000}
\pgfsetstrokecolor{dialinecolor}
\node[anchor=west] at (8.000000\du,5.500000\du){$x_2$};
\pgfsetlinewidth{0.050000\du}
\pgfsetdash{{\pgflinewidth}{0.200000\du}}{0cm}
\pgfsetdash{{\pgflinewidth}{0.040000\du}}{0cm}
\pgfsetmiterjoin
\pgfsetbuttcap
{
\definecolor{dialinecolor}{rgb}{0.000000, 0.000000, 0.000000}
\pgfsetfillcolor{dialinecolor}
% was here!!!
\definecolor{dialinecolor}{rgb}{0.000000, 0.000000, 0.000000}
\pgfsetstrokecolor{dialinecolor}
\pgfpathmoveto{\pgfpoint{13.750000\du}{4.517860\du}}
\pgfpathcurveto{\pgfpoint{13.750000\du}{4.517860\du}}{\pgfpoint{13.750000\du}{4.517860\du}}{\pgfpoint{13.750000\du}{4.517860\du}}
\pgfusepath{stroke}
}
\pgfsetlinewidth{0.050000\du}
\pgfsetdash{{\pgflinewidth}{0.040000\du}}{0cm}
\pgfsetdash{{\pgflinewidth}{0.040000\du}}{0cm}
\pgfsetmiterjoin
\pgfsetbuttcap
{
\definecolor{dialinecolor}{rgb}{0.000000, 0.000000, 0.000000}
\pgfsetfillcolor{dialinecolor}
% was here!!!
\definecolor{dialinecolor}{rgb}{0.000000, 0.000000, 0.000000}
\pgfsetstrokecolor{dialinecolor}
\pgfpathmoveto{\pgfpoint{13.750000\du}{4.517860\du}}
\pgfpathcurveto{\pgfpoint{13.750000\du}{4.517860\du}}{\pgfpoint{13.750000\du}{4.517860\du}}{\pgfpoint{13.750000\du}{4.517860\du}}
\pgfusepath{stroke}
}
\pgfsetlinewidth{0.050000\du}
\pgfsetdash{{\pgflinewidth}{0.040000\du}}{0cm}
\pgfsetdash{{\pgflinewidth}{0.200000\du}}{0cm}
\pgfsetmiterjoin
\pgfsetbuttcap
{
\definecolor{dialinecolor}{rgb}{0.000000, 0.000000, 0.000000}
\pgfsetfillcolor{dialinecolor}
% was here!!!
\definecolor{dialinecolor}{rgb}{0.000000, 0.000000, 0.000000}
\pgfsetstrokecolor{dialinecolor}
\pgfpathmoveto{\pgfpoint{12.750000\du}{3.250000\du}}
\pgfpathcurveto{\pgfpoint{13.500000\du}{2.750000\du}}{\pgfpoint{14.250000\du}{3.750000\du}}{\pgfpoint{14.750000\du}{3.250000\du}}
\pgfusepath{stroke}
}
\pgfsetlinewidth{0.050000\du}
\pgfsetdash{{\pgflinewidth}{0.200000\du}}{0cm}
\pgfsetdash{{\pgflinewidth}{0.200000\du}}{0cm}
\pgfsetmiterjoin
\pgfsetbuttcap
{
\definecolor{dialinecolor}{rgb}{0.000000, 0.000000, 0.000000}
\pgfsetfillcolor{dialinecolor}
% was here!!!
\definecolor{dialinecolor}{rgb}{0.000000, 0.000000, 0.000000}
\pgfsetstrokecolor{dialinecolor}
\pgfpathmoveto{\pgfpoint{16.150000\du}{6.050000\du}}
\pgfpathcurveto{\pgfpoint{16.900000\du}{5.550000\du}}{\pgfpoint{18.150000\du}{6.550000\du}}{\pgfpoint{18.650000\du}{6.050000\du}}
\pgfusepath{stroke}
}
\pgfsetlinewidth{0.100000\du}
\pgfsetdash{}{0pt}
\pgfsetdash{}{0pt}
\pgfsetmiterjoin
\pgfsetbuttcap
{
\definecolor{dialinecolor}{rgb}{0.000000, 0.000000, 0.000000}
\pgfsetfillcolor{dialinecolor}
% was here!!!
\pgfsetarrowsend{stealth}
{\pgfsetcornersarced{\pgfpoint{0.000000\du}{0.000000\du}}\definecolor{dialinecolor}{rgb}{0.000000, 0.000000, 0.000000}
\pgfsetstrokecolor{dialinecolor}
\draw (16.150000\du,8.275000\du)--(1.725000\du,8.275000\du)--(1.725000\du,6.750000\du)--(3.775000\du,6.750000\du)--(3.775000\du,6.750000\du);
}}
\end{tikzpicture}
}  
  \caption[Level and flow rig photograph and flow diagram]{Level and flow rig photograph (left) and process flow diagram (right).}
  \label{fig:flowphoto}
\end{figure}

\subsubsection{Process model}
This rig is modelled as a $2\times2$ system.
The MVs are the milliamp signals sent to the valves ($x_1\text{ \& }x_2$) and correspond to the fractional openings of the two valves.
The CVs are the flow ($F$) and the level ($L$).
The steady-state gain matrix of this process ($G_{fl}$) is shown in equation~\ref{eq:flowmodel}.
Note that action of valve 1 is air-to-open and valve 2 is air-to-close valve, which explains the signs of the gains in the first column of $G_{fl}$.
\begin{equation}
  \label{eq:flowmodel}
                    %x1       x2
    G_{fl} = \bpm -0.0476 & -0.0498 \\      %L
                  0.0111 & -0.0604 \\ \epm %F
\end{equation}
\subsubsection{Operating conditions}
The operating conditions for this rig are shown in table~\ref{tab:flowopcon}
The nominal operating point (output space) is; 1.322 gpm ($F$) and 16.4048 cm ($L$).
\begin{table}[htbp]
  \centering
  \begin{tabular}{llllll}
    \toprule
    \multicolumn{2}{c}{Variable} & \multicolumn{2}{c}{Operational constraints} & \multicolumn{2}{c}{Physical constraints} \\
    && Low & High & Low & High \\ 
    \midrule
    Inputs &$x_1 (\text{mA})$ & 4 & 20 & 4 & 20 \\
           &$x_2 (\text{mA})$ & 4 & 20 & 4 & 20 \\[1.3ex]
    Outputs &$F~(\text{gpm})$ & 1.2 & 1.5 & 0 & 1.9 \\
            &$L~(\text{cm})$  & 15 & 18 & 6 & 25 \\
    \bottomrule
  \end{tabular}
  \caption{Operating conditions of level and flow rig.}
  \label{tab:flowopcon}
\end{table}

\subsection{Laboratory distillation column}
\subsubsection{System description}
The second case study investigated is a 10-plate distillation column.
The column is run as a closed system with bottoms and distillate being mixed and fed back into the column.
Figure~\ref{fig:columnphoto} shows a photograph of the column along with its process flow diagram.
\begin{figure}[htbp]
  \centering
    \includegraphics[width=7.4cm]{graph/columnphoto.jpg}
    %\qquad
%    \includegraphics[width=7.8cm]{graph/columnpfd}
    \scalebox{1}{\input{graph/columnpfd}}  
  \caption[Laboratory distillation column photograph and flow diagram]{Laboratory distillation column photograph (left) and flow diagram (right).}
  \label{fig:columnphoto}
\end{figure}

\subsubsection{System model}
The process model has been reduced to a $2\times2$ matrix.
Reflux flowrate ($R$, expressed as a milliamp value sent to the valve) and the setpoint of plate 10's temperature ($T_{10~sp}$) are the MVs.
The temperatures of plate 1 and 8 ($T_1\text{ \& }T_{8}$) are the CVs.
Figure~\ref{fig:columnmodel} shows a graphic representation of the column model and equation~\ref{eq:columnmodel} gives the values for the steady-state gain model.
The nominal operating point (output space) is; 68~\textcelsius ($T_1$) and 78~\textcelsius ($T_8$).
\begin{equation}
  \label{eq:columnmodel}
                 % R      T10sp
  G_{col} = \bpm -0.0575 & 0.96 \\       % T1
                -0.146  & 0.518 \\ \epm % T8
\end{equation}
\begin{figure}[htbp]
  \centering
%    \includegraphics[width=8cm]{graph/columnmodel}
    \scalebox{1}{% Graphic for TeX using PGF
% Title: /home/andre/GIT Repos/AHCampher_thesis/diagrams/columnmodel.dia
% Creator: Dia v0.97.1
% CreationDate: Thu Jan  6 17:32:45 2011
% For: andre
% \usepackage{tikz}
% The following commands are not supported in PSTricks at present
% We define them conditionally, so when they are implemented,
% this pgf file will use them.
\ifx\du\undefined
  \newlength{\du}
\fi
\setlength{\du}{15\unitlength}
\begin{tikzpicture}
\pgftransformxscale{1.000000}
\pgftransformyscale{-1.000000}
\definecolor{dialinecolor}{rgb}{0.000000, 0.000000, 0.000000}
\pgfsetstrokecolor{dialinecolor}
\definecolor{dialinecolor}{rgb}{1.000000, 1.000000, 1.000000}
\pgfsetfillcolor{dialinecolor}
\pgfsetlinewidth{0.100000\du}
\pgfsetdash{{1.000000\du}{1.000000\du}}{0\du}
\pgfsetdash{{0.200000\du}{0.200000\du}}{0\du}
\pgfsetmiterjoin
\definecolor{dialinecolor}{rgb}{1.000000, 1.000000, 1.000000}
\pgfsetfillcolor{dialinecolor}
\fill (1.000000\du,4.500000\du)--(1.000000\du,12.000000\du)--(11.500000\du,12.000000\du)--(11.500000\du,4.500000\du)--cycle;
\definecolor{dialinecolor}{rgb}{0.000000, 0.000000, 0.000000}
\pgfsetstrokecolor{dialinecolor}
\draw (1.000000\du,4.500000\du)--(1.000000\du,12.000000\du)--(11.500000\du,12.000000\du)--(11.500000\du,4.500000\du)--cycle;
\pgfsetlinewidth{0.100000\du}
\pgfsetdash{}{0pt}
\pgfsetdash{}{0pt}
\pgfsetmiterjoin
\definecolor{dialinecolor}{rgb}{1.000000, 1.000000, 1.000000}
\pgfsetfillcolor{dialinecolor}
\fill (6.000000\du,5.500000\du)--(6.000000\du,11.000000\du)--(10.500000\du,11.000000\du)--(10.500000\du,5.500000\du)--cycle;
\definecolor{dialinecolor}{rgb}{0.000000, 0.000000, 0.000000}
\pgfsetstrokecolor{dialinecolor}
\draw (6.000000\du,5.500000\du)--(6.000000\du,11.000000\du)--(10.500000\du,11.000000\du)--(10.500000\du,5.500000\du)--cycle;
% setfont left to latex
\definecolor{dialinecolor}{rgb}{0.000000, 0.000000, 0.000000}
\pgfsetstrokecolor{dialinecolor}
\node[anchor=west] at (6.500000\du,8.2500000\du){Column};
\pgfsetlinewidth{0.100000\du}
\pgfsetdash{}{0pt}
\pgfsetdash{}{0pt}
\pgfsetbuttcap
{
\definecolor{dialinecolor}{rgb}{0.000000, 0.000000, 0.000000}
\pgfsetfillcolor{dialinecolor}
% was here!!!
\pgfsetarrowsend{stealth}
\definecolor{dialinecolor}{rgb}{0.000000, 0.000000, 0.000000}
\pgfsetstrokecolor{dialinecolor}
\draw (0.000000\du,6.500000\du)--(6.000000\du,6.500000\du);
}
\pgfsetlinewidth{0.100000\du}
\pgfsetdash{}{0pt}
\pgfsetdash{}{0pt}
\pgfsetbuttcap
{
\definecolor{dialinecolor}{rgb}{0.000000, 0.000000, 0.000000}
\pgfsetfillcolor{dialinecolor}
% was here!!!
\pgfsetarrowsend{stealth}
\definecolor{dialinecolor}{rgb}{0.000000, 0.000000, 0.000000}
\pgfsetstrokecolor{dialinecolor}
\draw (0.000000\du,10.000000\du)--(2.000000\du,10.000000\du);
}
\pgfsetlinewidth{0.100000\du}
\pgfsetdash{}{0pt}
\pgfsetdash{}{0pt}
\pgfsetmiterjoin
\definecolor{dialinecolor}{rgb}{1.000000, 1.000000, 1.000000}
\pgfsetfillcolor{dialinecolor}
\fill (2.000000\du,9.000000\du)--(2.000000\du,11.000000\du)--(4.000000\du,11.000000\du)--(4.000000\du,9.000000\du)--cycle;
\definecolor{dialinecolor}{rgb}{0.000000, 0.000000, 0.000000}
\pgfsetstrokecolor{dialinecolor}
\draw (2.000000\du,9.000000\du)--(2.000000\du,11.000000\du)--(4.000000\du,11.000000\du)--(4.000000\du,9.000000\du)--cycle;
\pgfsetlinewidth{0.100000\du}
\pgfsetdash{}{0pt}
\pgfsetdash{}{0pt}
\pgfsetbuttcap
{
\definecolor{dialinecolor}{rgb}{0.000000, 0.000000, 0.000000}
\pgfsetfillcolor{dialinecolor}
% was here!!!
\pgfsetarrowsend{stealth}
\definecolor{dialinecolor}{rgb}{0.000000, 0.000000, 0.000000}
\pgfsetstrokecolor{dialinecolor}
\draw (4.000000\du,10.000000\du)--(6.000000\du,10.000000\du);
}
% setfont left to latex
\definecolor{dialinecolor}{rgb}{0.000000, 0.000000, 0.000000}
\pgfsetstrokecolor{dialinecolor}
\node[anchor=west] at (2.2500000\du,10.000000\du){PI};
\node[anchor=west] at (4.2500000\du,10.750000\du){$Q$};
\pgfsetlinewidth{0.100000\du}
\pgfsetdash{}{0pt}
\pgfsetdash{}{0pt}
\pgfsetbuttcap
{
\definecolor{dialinecolor}{rgb}{0.000000, 0.000000, 0.000000}
\pgfsetfillcolor{dialinecolor}
% was here!!!
\pgfsetarrowsend{stealth}
\definecolor{dialinecolor}{rgb}{0.000000, 0.000000, 0.000000}
\pgfsetstrokecolor{dialinecolor}
\draw (10.500000\du,6.500000\du)--(16.500000\du,6.500000\du);
}
\pgfsetlinewidth{0.100000\du}
\pgfsetdash{}{0pt}
\pgfsetdash{}{0pt}
\pgfsetbuttcap
{
\definecolor{dialinecolor}{rgb}{0.000000, 0.000000, 0.000000}
\pgfsetfillcolor{dialinecolor}
% was here!!!
\pgfsetarrowsend{stealth}
\definecolor{dialinecolor}{rgb}{0.000000, 0.000000, 0.000000}
\pgfsetstrokecolor{dialinecolor}
\draw (10.500000\du,10.000000\du)--(16.500000\du,10.000000\du);
}
% setfont left to latex
\definecolor{dialinecolor}{rgb}{0.000000, 0.000000, 0.000000}
\pgfsetstrokecolor{dialinecolor}
\node[anchor=west] at (11.500000\du,12.500000\du){$G_{col}$};
% setfont left to latex
\definecolor{dialinecolor}{rgb}{0.000000, 0.000000, 0.000000}
\pgfsetstrokecolor{dialinecolor}
\node[anchor=west] at (-1.250000\du,6.500000\du){$R$};
% setfont left to latex
\definecolor{dialinecolor}{rgb}{0.000000, 0.000000, 0.000000}
\pgfsetstrokecolor{dialinecolor}
\node[anchor=west] at (-2.000000\du,10.250000\du){$T_{10~sp}$};
% setfont left to latex
\definecolor{dialinecolor}{rgb}{0.000000, 0.000000, 0.000000}
\pgfsetstrokecolor{dialinecolor}
\node[anchor=west] at (16.500000\du,6.500000\du){$T_1$};
% setfont left to latex
\definecolor{dialinecolor}{rgb}{0.000000, 0.000000, 0.000000}
\pgfsetstrokecolor{dialinecolor}
\node[anchor=west] at (16.500000\du,10.000000\du){$T_8$};
\end{tikzpicture}
}  
  \caption[Column model]{Column model showing internal PI controller.}
  \label{fig:columnmodel}
\end{figure}

The PI controller between $T_{10~sp}$ and $Q$ is contained in the process' baselayer. 
This was added as a safety measure, should the advanced control layer fail.

\subsubsection{Operating conditions}
Table~\ref{tab:columnopcon} shows the operating conditions of the distillation column.
\begin{table}[htbp]
  \centering
  \begin{tabular}{llllll}
    \toprule
    \multicolumn{2}{c}{Variable} & \multicolumn{2}{c}{Operational constraints} & \multicolumn{2}{c}{Physical constraints} \\
    && Low & High & Low & High \\ 
    \midrule
    Inputs &$R~(\text{mA})$          & 11 & 15 & 4 & 20 \\
           &$T_{10~sp}~(\textcelsius)$ & 78 & 82 & 0 & 90 \\[1.3ex]
    Outputs &$T_1~(\textcelsius)$     & 66 & 68 & 15 & 68 \\
            &$T_{8}~(\textcelsius)$   & 78 & 82 & 25 & 90 \\
    \bottomrule
  \end{tabular}
  \caption{Operating conditions of distillation column.}
  \label{tab:columnopcon}
\end{table}

% Local Variables:
% TeX-master: "AHC_thesis"
% End:
