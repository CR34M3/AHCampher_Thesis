\chapter{Introduction}\label{chap:intro}
\section{Background}
Model predictive controllers (MPCs) use models to predict the future behaviour of a process.
The ability of some MPC algorithms to impose constraints on the outputs or inputs of a system is their biggest selling point.

Just as the process predicts the effect of the inputs on the outputs, the constraints imposed on the system are also interdependent (via the model).
At present, commercial MPC packages do not validate constraints based on the process model.
This gives rise to the following problems;
\begin{itemize}
  \item Specified constraints on an input/output may be infeasible due to their corresponding output/input requirements.
  \item Setpoint specification may be infeasible as the process has a limited output space (due to the input constraints).
\end{itemize}

A systematic approach to constraint management would therefore be an attractive addition to any MPC package. 
Such a method would help to avoid infeasible constraints and further the understanding of constraint interaction in MPCs.

\section{Problem statement}
The objective is therefore to develop a method to systematically manage constraints imposed on MPCs.
Mathematical rigour and unambiguous language when specifying constraints were key design criteria.

\section{Method}
The method builds on the work by \citet{vinsonphd}, \citet{limaphd} and \citet{opconproc} in interpreting constraint interactions.

The routines for this project were developed using the Python programming language.
They were tested for functionality on the models and operating conditions of laboratory rigs.

\section{Scope and deliverables}
For this project, linear steady-state models are considered.
The constraints imposed on the systems are also assumed to be linear.
With these assumptions made, the project can now focus exclusively on convex sets in the input and output space.

It needs to be noted that this project does not attempt to reinvent or propose an alternative to the method of internal constraint handling of MPC algorithms.

The deliverables of this project are;
\begin{itemize}
\item Routines to perform a systematic analysis of constraints based on the process model.
\item Results indicating the use of the aforementioned routines on practical systems.
\end{itemize}

% Local Variables:
% TeX-master: "AHC_thesis"
% End: