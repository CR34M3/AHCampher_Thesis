\chapter{Introduction}\label{chap:intro}
\section{Background}
Model predictive controllers (MPCs) use models to predict the future behaviour of a process.
The ability of some MPC algorithms to impose constraints on the outputs or inputs of a system is their biggest selling point.

In the same way the process model predicts the effect of the inputs on the outputs, the process model dictates the interdependence of the constraints imposed on the system.
At present, commercial MPC packages do not validate constraints based on the process model.
This gives rise to the following problems:
\begin{itemize}
  \item Specified constraints on an input or output may be infeasible due to their corresponding output or input requirements.
  \item Setpoint specification may be infeasible as the process has a limited output space (due to the input constraints).
\end{itemize}

A systematic approach to constraint management would therefore be an attractive addition to any MPC package. 
Such a method would help to avoid infeasible constraints and further the understanding of constraint interaction in MPCs.

\section{Problem statement}
The objective is therefore to develop systematic method of managing constraints imposed on MPCs.
Mathematical rigour and disambiguation of the language used to specify constraints were key design criteria.

\section{Method}
The method builds on the work by \citet{vinsonphd}, \citet{limaphd} and \citet{opconproc} in interpreting constraint interactions.

The routines for this project were developed using the Python \citep{pythonref} programming language.
Version management of the source code was done using git \citep{gitref} and hosted on github \citep{githubref, githubrefcode, githubrefdiss}.
This allowed for continuous and collaborative development and flagging of issues.

The routines were tested for functionality on the models and operating conditions of laboratory rigs.

\section{Scope and deliverables}
For this project, linear steady-state models are considered.
The constraints imposed on the systems are also assumed to be linear.
With these assumptions made, the project can now focus exclusively on convex sets in the input and output space.

This project does not attempt to reinvent or propose an alternative to the method of internal constraint handling of MPC algorithms.
The method aims to provide a supportive framework in both the controller design and operation phases.

The deliverables of this project are;
\begin{itemize}
\item A framework for systematic constraint handling for MPCs.
\item Routines to perform an analysis of constraints based on interaction via the process model.
\item Results indicating the use of the aforementioned routines on practical systems.
\end{itemize}

% Local Variables:
% TeX-master: "AHC_thesis"
% End:
