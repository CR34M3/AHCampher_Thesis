\chapter*{Synopsis}\addcontentsline{toc}{section}{Synopsis}
The models used by model predictive controllers (MPCs) to predict future outcomes are usually unconstrained forms like impulse or step responses and discrete state space models. 
Certain MPC algorithms allow constraints  to be imposed on the inputs or outputs of a system; but they may be infeasible as they are not checked for consistency via the process model. 
Consistent constraint handling methods -- which account for their interdependence and disambiguate the language used to specify constraints -- would therefore be an attractive addition to any MPC package.

A rigorous and systematic approach to constraint management has been developed, building on the work of Georgakis and others in interpreting constraint interactions. 
The method supports linear and non-linear (polynomial) steady-state system models, and provides an interface where the following information can be obtained;
\begin{itemize}
  \item effects of constraint changes on the corresponding input/output constraints,
  \item feasibility checks for constraints,
  \item constraint-type information,
  \item specification of constraint-set size and
  \item optimal fitting of constraints within the desirable input/output space.
\end{itemize}
Mathematical rigour and unambiguous language for identifying constraint types were key design criteria. 
Ample feedback to the user was added to provide a supportive rather than prescriptive environment.

The outputs of the program are compatible with commercial MPC packages, such as Honeywell’s RMPCT$^{\copyright}$ and AspenTech’s DMCPlus$^{\copyright}$.
These packages were used in conjunction with the developed software to test functionality and performance of the method.
The method was applied to case studies from Anglo Platinum, the Tennessee Eastman sample problem and laboratory scale test rigs.
\bigskip

\noindent \textbf{KEYWORDS:} constraint handling, model predictive
controllers, geometric methods, operability index

% Local Variables:
% TeX-master: "AHC_thesis"
% End: