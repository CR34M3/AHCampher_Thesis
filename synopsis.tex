\begin{center} \textbf{\Large A Systematic Approach to Model Predictive Controller Constraint Handling: Rigorous Geometric Methods}
\end{center}
\bigskip
By: \textbf{Andr\'e Herman Campher}\\
Study leader: \textbf{Carl Sandrock}\\
Department: \textbf{Chemical Engineering}\\
Degree: \textbf{MEng Control Engineering}
\vfill
\begin{center} \Large{SYNOPSIS}\addcontentsline{toc}{section}{Synopsis}
\end{center}
\vfill
The models used by model predictive controllers (MPCs) to predict future outcomes are usually unconstrained forms like impulse or step responses and discrete state space models. 
Certain MPC algorithms allow constraints  to be imposed on the inputs or outputs of a system; but they may be infeasible as they are not checked for consistency via the process model. 
Consistent constraint handling methods -- which account for their interdependence and disambiguate the language used to specify constraints -- would therefore be an attractive aid when using any MPC package.

A rigorous and systematic approach to constraint management has been developed, building on the work of \citet{vinsonphd}, \citet{limaphd} and \citet{opconproc} in interpreting constraint interactions. 
The method supports linear steady-state system models, and provides routines to obtain the following information:
\begin{itemize}
  \item effects of constraint changes on the corresponding input and output constraints,
  \item feasibility checks for constraints,
  \item specification of constraint-set size and
  \item optimal fitting of constraints within the desirable input and output space.
\end{itemize}
Mathematical rigour and unambiguous language for identifying constraint types were key design criteria. 

The outputs of the program provide guidance when handling constraints, as opposed to rules of thumb and experience, and promote understanding of the system and its constraints.
The metrics presented are not specific to any commercial MPC and can be implemented in the user interfaces of such MPCs.
The method was applied to laboratory-scale test rigs to illustrate the information obtained.
\vfill

\noindent \textbf{KEYWORDS:} constraint handling, model predictive
controllers, geometric methods, operability index

% Local Variables:
% TeX-master: "AHC_thesis"
% End: